\chapter{Introdução}

%=====================================================

% A introdução geral do documento pode ser apresentada através das seguintes seções: Desafio, Motivação, Proposta, Contribuição e Organização do documento (especificando o que será tratado em cada um dos capítulos). O Capítulo 1 não contém subseções\footnote{Ver o Capítulo \ref{cap-exemplos} para comentários e exemplos de subseções.}.

O termo microrrede descreve o conjunto de microfontes, cargas e sistemas de armazenamento de energia elétrica que operam em conjunto e através de um sistema único de controle. Dessa forma, pode oferecer para seu usuário uma melhor confiabilidade na qualidade da energia, possibilidade da operação ilhada ou completamente desconectada da rede de distribuição \cite{Las02,Tan13}.

De forma prática, uma microrrede é composta por diversos tipos de fontes de energia. Mais especificamente, fontes caracterizadas como Geração Distribuída (GD). Tipicamente pequenas turbinas (aerogeradores de pequeno porte ou pequenas centrais hidrelétricas (PCHs)) painéis fotovoltaicos ou células a combustível de baixo custo, baixa tensão e baixa emissão de poluentes. 

Sistemas de armazenamento de energia (ESS) são indispensáveis em microrredes com fontes de energia renováveis, visto que esse tipo de fonte não tem a habilidade de responder a uma demanda imediata, sendo intermitentes e dependentes de variações sazonais. Além disso, ESSs proporcionam um aumento na confiabilidade e qualidade da energia elétrica \cite{Zob18}.

Além disso, microrredes ainda são compostas de cargas diversificadas e conversores estáticos (CA-CC, CC-CC e CC-CA). Com eles, o fluxo de potência entre os dois barramentos pode ser facilmente controlado.

Veículos elétricos são muito parecidos com microrredes, com os 3 principais componentes: geradores, armazenamento de energia e conversores. Dependendo do nível de eletrificação (híbrido, híbrido plug-in ou elétrico), um motor a combustão é usado como gerador de energia elétrica. As baterias têm capacidade e potência para suprir energia aos motores, sendo que este é controlado por um inversor (conversor de energia CC-CA).

Neste trabalho serão investigados os principais aspectos no desenvolvimento de projetos de sistemas de armazenamento de energia para microrredes e veículos elétricos, buscando o melhor desempenho e segurança na operação. 

%=====================================================

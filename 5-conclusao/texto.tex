\chapter{Conclusões}
\label{cap:exemplos}

%=====================================================

O principal objetivo desse trabalho foi efetuar uma pesquisa sobre o uso de sistemas de armazenamento de energia em microrredes e veículos elétricos para o desenvolvimento de um projeto completo e seguro.

Os sistemas da microrrede e do veículo elétrico foram estudados, além disso, também os principais componentes tiveram seu funcionamento investigado. Com isso foi possível verificar requisitos dos sistemas que integrariam diretamente com as baterias. Várias tecnologias de sistemas de armazenamento de energia foram estudadas, com isso foi possível ter um panorama geral do mercado e dos usos atuais e futuros.

Verificou-se que a tecnologia de baterias de íons de lítio é atualmente a que apresenta as maiores densidades de energia e potência, sendo por isso escolhida para o uso nesse projeto. Seu funcionamento foi estudado, assim como todos os aspectos de segurança, para definir então a melhor forma de operação e requisitos de projeto.

Foi definida uma metodologia de projetos, que a partir daí foi desenvolvido. O sistema de armazenamento foi separado em vários subsistemas e requerimentos individuais foram definidos. Foi desenvolvido o projeto de arquitetura e integração dos sistemas, para mostrar como cada parte seria integrada no projeto, assim como prover documentação para integração externa. Na fase de especificação dos componentes, estes foram definidos de forma a proporcionar uma descrição completa para as compras e manufatura dos sistemas.

O presente trabalho é uma referência para futuros trabalhos dentro da Universidade Federal do Paraná e da equipe UFPR Formula, para que seja feita uma correta operação e manutenção dos sistemas. Para a finalização, ainda é necessário que a fase de manufatura seja realizada, que foi atrasada pela pandemia de COVID-19, além de todos os testes de componente, integração e sistema, tanto do armazenamento para a microrrede quanto do veículo elétrico.

% Pacotes usados neste documento e suas respectivas configurações

% seleção de línguas do texto (a última é a principal/default)
\usepackage[english,brazilian]{babel}

% ------------------------------------------------------------------------------
% Definição de fontes

% formato dos arquivos-fonte (utf8 no Linux e latin1 no Windows)
\usepackage[utf8]{inputenc}	% arquivos LaTeX em Unicode (UTF8)

% usar codificação T1 para ter caracteres acentuados corretos no PDF
\usepackage[T1]{fontenc}

% fonte usada no corpo do texto (descomente apenas uma)
\usepackage{newtxtext,newtxmath}	% Times (se não tiver, use mathptmx)
%\usepackage{lmodern}			% Computer Modern (fonte clássico LaTeX)
%\usepackage{kpfonts}			% Kepler/Palatino (idem, use mathpazo)
%\renewcommand{\familydefault}{\sfdefault} % Arial/Helvética (leia abaixo)

% A biblioteca central da UFPR recomenda usar Arial, seguindo a recomendação da
% ABNT. Essa é uma escolha ruim, pois fontes sans-serif são geralmente inade-
% quados para textos longos e impressos, sendo melhores para páginas Web.
% http://www.webdesignerdepot.com/2013/03/serif-vs-sans-the-final-battle/.

% fontes usadas em ambientes específicos
\usepackage[scaled=0.9]{helvet}		% Sans Serif
\usepackage{courier}			% Verbatim, Listings, etc

% ------------------------------------------------------------------------------

% inclusão de figuras
\usepackage{graphicx}			% incluir figuras em PDF, PNG, PS, EPS

% subfiguras (subfigure is deprecated, don't use it)
\usepackage[labelformat=simple]{subcaption}
\renewcommand\thesubfigure{(\alph{subfigure})}

% ------------------------------------------------------------------------------

% inclusão/formatação de código-fonte (programas)
\usepackage{listings}
\lstset{language=c}
\lstset{basicstyle=\ttfamily\footnotesize,commentstyle=\textit,stringstyle=\ttfamily}
\lstset{showspaces=false,showtabs=false,showstringspaces=false}
\lstset{numbers=left,stepnumber=1,numberstyle=\tiny}
\lstset{columns=flexible,mathescape=true}
\lstset{frame=single}
\lstset{inputencoding=utf8,extendedchars=true}
\lstset{literate={á}{{\'a}}1  {ã}{{\~a}}1 {à}{{\`a}}1 {â}{{\^a}}1
                 {Á}{{\'A}}1  {Ã}{{\~A}}1 {À}{{\`A}}1 {Â}{{\^A}}1
                 {é}{{\'e}}1  {ê}{{\^e}}1 {É}{{\'E}}1  {Ê}{{\^E}}1
                 {í}{{\'\i}}1 {Í}{{\'I}}1
                 {ó}{{\'o}}1  {õ}{{\~o}}1 {ô}{{\^o}}1
                 {Ó}{{\'O}}1  {Õ}{{\~O}}1 {Ô}{{\^O}}1
                 {ú}{{\'u}}1  {Ú}{{\'U}}1
                 {ç}{{\c{c}}}1 {Ç}{{\c{C}}}1 }

% formatação de algoritmos
\usepackage{algorithm,algorithmic}
\floatname{algorithm}{Algoritmo}
\renewcommand{\algorithmiccomment}[1]{~~~// #1}
%\algsetup{linenosize=\footnotesize,linenodelimiter=.}

% ------------------------------------------------------------------------------

% outros pacotes
\usepackage{alltt,moreverb}	% mais comandos no modo verbatim
\usepackage{lipsum}		% gera texto aleatório (para os exemplos)
\usepackage{currfile}		% infos sobre o arquivo/diretório atual
\usepackage[final]{pdfpages}	% inclusão de páginas em PDF
\usepackage{longtable}		% tabelas multi-páginas (tab símbolos/acrônimos)

% listas de símbolos e de abreviações (a fazer)
%\usepackage[titles]{tocloft}
%\newlistof[part]{symb}{los}{Lista de Símbolos}
%\newlistof[part]{abbrev}{loa}{Lista de Abreviações}
%\newcommand{\symb}[2]{%
%\refstepcounter{symb}
%\addcontentsline{los}{symb}{\protect #1 :#2}\par}


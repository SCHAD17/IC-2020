\begin{resumo}

Uma crescente preocupação com fatores climáticos e com o aquecimento global visando uma diminuição do consumo de combustíveis fósseis, vem abrindo espaço para o desenvolvimento de novos modelos na geração e consumo de energia. Uma evolução das fontes renováveis de energia trouxe a possibilidade de uma geração descentralizada, surgindo assim, o conceito da microrrede. Essas, trazem novas necessidades na manutenção da qualidade e no controle da energia elétrica em redes com geração intermitente, levando assim, ao uso de sistemas de armazenamento de energia. Além disso, o avanço na mobilidade elétrica, proporcionando uma maior densidade de energia das baterias, possibilitou o surgimento de veículos elétricos baratos e competitivos.

Desta forma, o objetivo deste trabalho foi o desenvolvimento de dois sistemas de armazenamento de energia, um conectado à microrrede do Departamento de Engenharia Elétrica (DELT) localizado no Centro Politécnico da Universidade Federal do Paraná (UFPR) em Curitiba e outra utilizada em um veículo elétrico do tipo Formula SAE, para competição nacional pela equipe UFPR Formula.

Primeiramente foi feito um estudo dos sistemas nos quais o armazenamento seria integrado, com uma revisão teórica do funcionamento dos principais componentes; foi então feito um estudo dos diferentes tipos e tecnologias de sistemas de armazenamento de energia, suas aplicações, características, estado da arte e aspectos de segurança; uma metodologia de organização do projeto foi definida: inicialmente definidos requerimentos para cada subsistema, depois projeto da arquitetura e integração do sistema, chegando na especificação do componente; e finalmente, como resultado se teve um projeto completo e especificado para os dois sistemas.

\end{resumo}
